% !TeX program = pdflatex
% !BIB program = bibtex

\documentclass{article}
\usepackage[margin=1in]{geometry}

\usepackage{tikz}
\usetikzlibrary{dsp,chains}

\usepackage{cleveref}

\title{Non-Uniform Filterbanks for Digital Audio Effects}
\author{Jatin Chowdhury}
\date{\today}

\begin{document}
\maketitle

\section{Abstract}
This project will explore the possibilities of building
audio effects based on non-uniform filter banks. The end goal is to
produce four effects in the form of audio plugins:
\begin{enumerate}
    \item Multiband dynamics meter: A metering plugin that splits the
    spectrum into 10 critical frequency bands. The plugin will show the
    peak and RMS gain of each band in real time. This sort of plugin will
    be useful for mixing and mastering engineers in analyzing the dynamic
    range of a song in individual frequency bands.

    \item Mastering Graphic EQ: In mastering, it is often desireable
    to use linear phase filters for equalization with reduced phase
    distortion. Most graphic equalizers, on the other hand, introduce
    some amount of phase distortion, since they do not use perfect
    reconstruction (PR) filterbanks. The idea here is to create a
    ``mastering quality'' graphic EQ, through the implementation of a
    PR non-uniform filterbank. Note that unlike the metering plugin,
    this effect will require both analysis and synthesis filterbanks.

    \item Frequency ``Sound-A-Like'' effect: In mixing, it can sometimes be
    be useful to make two instruments sound nearly identical. To that end,
    this plugin will accept a ``sidechain'' input, and will filter the main
    input so that the frequency spectra of the two channels are identical.
    Note that this will require an analysis and synthesis filterbank for the
    main input, as well as an additional analysis filterbank for the sidechain
    input.

    \item Frequency ``Separator'' effect: A more common case in mixing is
    that the engineer has two tracks with similar frequency spectra, and
    wishes to ``separate'' the two tracks spectrally, so that the listener
    can distinguish more easily between the two. This plugin will function
    as an inverse to the ``Sound-A-Like'' plugin.
\end{enumerate}

A brief discussion of non-uniform filterbanking techniques is given below,
followed by signal flow diagrams for each individual plugin.

\section{Non-Uniform Fliterbank Techniques}
\begin{itemize}
    \item Polyphase filterbank \cite{SASPWEB2011}:
    For this method, a filterbank would be designed as a set of minimum-phase
    bandpass filters $H_k(z)$, as well as the set of inverse filters $F_k(z)$.
    Each analysis filter can then be deconstructed using a ``type 1'' polyphase
    representation:

    \begin{equation}
        H_k(z) = \sum_{l = 0}^{N-1} z^{-l} E_{kl}(z^N)
    \end{equation}

    And each synthesis filter can be deconstructed as ``type 2'' polyphase:

    \begin{equation}
        F_k(z) = \sum_{l = 0}^{N-1} z^{-N-l-1} R_{lk}(z^N)
    \end{equation}

    Using polyphase matrices $E(z)$ and $R(z)$, we can achieve perfect, in-phase
    reconstruction with reasonable efficiency. Note that it is necessary for
    the set of analysis filters to be minimum-phase to ensure
    that the inverse filters are causal and stable.
    
    In some implementations, such as \cite{GraphicEQ}, the polyphase method
    is used to create a uniform PR filterbank, then the channels are aggregated
    so as to generate a non-uniform filterbank. However, it should be possible
    to achieve greater efficiency by beginning with non-uniform filters.

    \item Constant-Q transform:
    Alternatively, a constant Q transform could be used to generate the filterbank.

    \begin{equation}
        X(k) = \frac{1}{N(k)} \sum_{n=0}^{N(k)-1} W(k,n) x(n) e^{\frac{-j2 \pi Qn}{N(k)}}
    \end{equation}

    where $W(k,n)$ is a windowing function, and
    \begin{equation}
        N(k) = Q \frac{f_s}{f_k}
    \end{equation}

    \cite{CQT_framework} gives a framework for invertible,
    real-time Constant-Q Transforms (CQTs), using non-stationary
    Gabor frames. \cite{Pitsh_Shift_CQT} discusses a specific real-time
    implementation of a 48 point invertible CQT, in the context of
    a pitch-shifting effect.
\end{itemize}

\section{Multiband Dynamics Meter}
A signal flow diagram for the Multiband Dynamics Meter can be seen in \cref{meter}.
Note that since this effect is a meter, no signal reconsntruction is required.
The visualizer will display both peak and RMS gains
for each channel.

\begin{figure*}[ht]
    \center
    \begin{tikzpicture}[
        node distance = 16mm and 10mm
                                ]
        \node (fb) [draw,minimum height=36mm] {Filter Bank};
        \coordinate[left = of fb.west] (x);
        \draw[dspconn] (x) node[left] {$x[n]$} -- (x-| fb.west);

        \coordinate[above right = 16mm and 10mm of fb.east] (t0);
        \coordinate [below = 7mm of t0] (t01);
        \coordinate[below = of t0] (t1);
        \coordinate [below = 7mm of t1] (t11);
        \coordinate[below = of t1] (t2);
        \coordinate [below = 7mm of t2] (t21);
        \draw[dspline] (fb.east |- t0) -- (t0);
        \draw[dspline] (t0) -- (t01);
        \draw[dspline] (fb.east |- t1) -- (t1);
        \draw[dspline] (t1) -- (t11);
        \draw[dspline] (fb.east |- t2) -- (t2);
        \draw[dspline] (t2) -- (t21);

        \node (p0) [draw, right = of t0] {Peak};
        \node (r0) [draw, below = 2mm of p0] {RMS};
        \node (p1) [draw, right = of t1] {Peak};
        \node (r1) [draw, below = 2mm of p1] {RMS};
        \node (p2) [draw, right = of t2] {Peak};
        \node (r2) [draw, below = 2mm of p2] {RMS};
        \draw[dspconn] (t0) -- (p0);
        \draw[dspconn] (t01) -- (r0);
        \draw[dspconn] (t1) -- (p1);
        \draw[dspconn] (t11) -- (r1);
        \draw[dspconn] (t2) -- (p2);
        \draw[dspconn] (t21) -- (r2);

        \node (vis) [draw,minimum height=48mm, above right= -30mm and 10mm of p1] {Visualizer};
        \draw[dspconn] (p0) -- (p0 -| vis.west);
        \draw[dspconn] (r0) -- (r0 -| vis.west);
        \draw[dspconn] (p1) -- (p1 -| vis.west);
        \draw[dspconn] (r1) -- (r1 -| vis.west);
        \draw[dspconn] (p2) -- (p2 -| vis.west);
        \draw[dspconn] (r2) -- (r2 -| vis.west);
    \end{tikzpicture}
    \caption{\label{meter}{\it Multiband Dynamics Meter signal flow, shown
                                   here with 3 channels.}}
\end{figure*}

\begin{figure*}[ht]
    \center
    \begin{tikzpicture}[
        node distance = 8mm and 10mm
                                ]
        \node (fb) [draw,minimum height=36mm] {Filter Bank};
        \coordinate[left = of fb.west] (x);
        \draw[dspconn] (x) node[left] {$x[n]$} -- (x-| fb.west);

        \node (a0) [draw, above right = 12mm and 10mm of fb.east] {Gain};
        \node (a1) [draw, below = of a0] {Gain};
        \node (a2) [draw, below = of a1] {Gain};
        \draw[dspconn] (fb.east |- a0) -- (a0);
        \draw[dspconn] (fb.east |- a1) -- (a1);
        \draw[dspconn] (fb.east |- a2) -- (a2);

        \node (rec) [draw,minimum height=36mm, above right= -22mm and 10mm of a1] {Reconstruct};
        \draw[dspconn] (a0) -- (a0 -| rec.west);
        \draw[dspconn] (a1) -- (a1 -| rec.west);
        \draw[dspconn] (a2) -- (a2 -| rec.west);

        \coordinate[above right = 3mm and 3mm of vis.east] (y);
        \draw[dspconn] (y -| rec.east) -- (y) node[right] {$y[n]$};
    \end{tikzpicture}
    \caption{\label{graphic}{\it Signal flow for Graphic EQ.}}
\end{figure*}

\section{Mastering Graphic EQ}
Signal flow for the Mastering Graphic EQ can be seen in
\cref{graphic}. Since the gain operators will only affect
the amplitude of each channel, if the filter
bank has perfect reconstruction, the output should not
introduce any phase distortion.

\section{Frequency ``Sound-A-Like'' and Separator}
The ``Sound-A-Like'' and Separator plugins will function similarly, with a detector
filter bank on the sidechain input (\cref{sidechain}), and an automatic gain control
filterbank, followed by a reconstruction filterbank on the main input (\cref{main}).
The detector output for each channel of the sidechain filterbank will
provide ``key'' the input for the automatic gain control on the corresponding
channel of the main filterbank. The only difference between the two plugins
will be the gain computers implemented in each plugin's automatic gain control
processors.

\begin{figure*}[ht]
    \center
    \begin{tikzpicture}[
        node distance = 8mm and 10mm
                                ]
        \node (fb) [draw,minimum height=36mm] {Filter Bank};
        \coordinate[left = of fb.west] (x);
        \draw[dspconn] (x) node[left] {$s[n]$} -- (x-| fb.west);

        \node (r0) [draw, above right = 12mm and 10mm of fb.east] {RMS};
        \node (r1) [draw, below = of r0] {RMS};
        \node (r2) [draw, below = of r1] {RMS};
        \draw[dspconn] (fb.east |- r0) -- (r0);
        \draw[dspconn] (fb.east |- r1) -- (r1);
        \draw[dspconn] (fb.east |- r2) -- (r2);
    \end{tikzpicture}
    \caption{\label{sidechain}{\it Signal flow for sidechain input.}}
\end{figure*}

\begin{figure*}[ht]
    \center
    \begin{tikzpicture}[
        node distance = 8mm and 10mm
                                ]
        \node (fb) [draw,minimum height=36mm] {Filter Bank};
        \coordinate[left = of fb.west] (x);
        \draw[dspconn] (x) node[left] {$x[n]$} -- (x-| fb.west);

        \node (a0) [draw, above right = 12mm and 10mm of fb.east] {AGC};
        \node (a1) [draw, below = of a0] {AGC};
        \node (a2) [draw, below = of a1] {AGC};
        \draw[dspconn] (fb.east |- a0) -- (a0);
        \draw[dspconn] (fb.east |- a1) -- (a1);
        \draw[dspconn] (fb.east |- a2) -- (a2);

        \node (rec) [draw,minimum height=36mm, above right= -22mm and 10mm of a1] {Reconstruct};
        \draw[dspconn] (a0) -- (a0 -| rec.west);
        \draw[dspconn] (a1) -- (a1 -| rec.west);
        \draw[dspconn] (a2) -- (a2 -| rec.west);

        \coordinate[above right = 3mm and 3mm of vis.east] (y);
        \draw[dspconn] (y -| rec.east) -- (y) node[right] {$y[n]$};
    \end{tikzpicture}
    \caption{\label{main}{\it Signal flow for main input.}}
\end{figure*}

%\newpage
\nocite{*}
\bibliographystyle{IEEEbib}
\bibliography{references}

\end{document}
